% \iffalse meta-comment
% 
% reqdoc.ins
% 
% Copyright (C) 2012 by Thai Son Hoang <htson at inf dot ethz dot ch>
% --------------------------------------------------------------------
% 
% This file may be distributed and/or modified under the
% conditions of the LaTeX Project Public License, either version 1.3
% of this license or (at your option) any later version.
% The latest version of this license is in:
% 
%      http://www.latex-project.org/lppl.txt
% 
% and version 1.3 or later is part of all distributions of LaTeX 
% version 2003/12/01 or later.
% 
% This work has the LPPL maintenance status "author-maintained".
% 
% This work consists of the files reqdoc.dtx, reqdoc.ins,
% the derived file reqdoc.sty, the generated documentation
% reqdoc.pdf, and some sample requirements documents.
% 
% \fi
% 
% \iffalse
%<reqdoc>\NeedsTeXFormat{LaTeX2e}\relax
%<reqdoc>\ProvidesPackage{reqdoc}
%<reqdoc>    [2012/06/29 v1.3 Macros for requirements documents] 
% 
%<*driver> 
\documentclass[a4paper]{ltxdoc}
\usepackage{reqdoc}
\EnableCrossrefs
% ^^A\CodelineIndex
\PageIndex
\RecordChanges

\begin{document}
\DocInput{reqdoc.dtx}
\end{document}
%</driver> 
% \fi
% 
% \CheckSum{253}
% 
% \CharacterTable
% {Upper-case    \A\B\C\D\E\F\G\H\I\J\K\L\M\N\O\P\Q\R\S\T\U\V\W\X\Y\Z
% Lower-case    \a\b\c\d\e\f\g\h\i\j\k\l\m\n\o\p\q\r\s\t\u\v\w\x\y\z
% Digits        \0\1\2\3\4\5\6\7\8\9
% Exclamation   \!     Double quote  \"     Hash (number) \#
% Dollar        \$     Percent       \%     Ampersand     \&
% Acute accent  \'     Left paren    \(     Right paren   \)
% Asterisk      \*     Plus          \+     Comma         \,
% Minus         \-     Point         \.     Solidus       \/
% Colon         \:     Semicolon     \;     Less than     \<
% Equals        \=     Greater than  \>     Question mark \?
% Commercial at \@     Left bracket  \[     Backslash     \\
% Right bracket \]     Circumflex    \^     Underscore    \_
% Grave accent  \`     Left brace    \{     Vertical bar  \|
% Right brace   \}     Tilde         \~}
% 
% 
% \changes{v1.0}{2010/10/28}{Initial version}
% \changes{v1.1}{2011/09/11}{Minor updates}
% \changes{v1.2}{2012/06/27}{Minor updates}
% \changes{v1.3}{2012/06/29}{Major updates}
% 
% \GetFileInfo{reqdoc.sty}
% 
% \DoNotIndex{\arabic}
% \DoNotIndex{\newcommand,\renewcommand,\newenvironment,\renewenvironment}
% \DoNotIndex{\newcounter,\value}
% \DoNotIndex{\\,\hline}
% \DoNotIndex{\begin,\end}
% \DoNotIndex{\item,\medskip,\quad,\small}
% \DoNotIndex{\label,\labelformat}
% \DoNotIndex{\DeclareOption,\ProcessOptions,\RequirePackage}
% \DoNotIndex{\ref,\refstepcounter,\setcounter}
% \DoNotIndex{\textsf,\textwidth}
%
% \title{The \textsf{reqdoc} package\thanks{This document
% corresponds to \textsf{reqdoc}~\fileversion, dated~\filedate.}}
% \author{Thai Son Hoang \\ ETH-Zurich \\ \texttt{<htson at inf dot
% ethz dot ch>}}
% \date{June 29, 2012}
% 
% \maketitle
% 
% ^^A %%%%% Abstract %%%%%
% \begin{abstract}
%   This package provides macros for typesetting requirements
%   documents.  It was developed at the Swiss Federal Institute of
%   Technology Zurich (ETH-Zurich).
% \end{abstract}
% 
% ^^A %%%%% Table of contents %%%%%
% \tableofcontents
% 
% ^^A %%%%% Introduction %%%%%
% \section{Introduction}
% 
% This package was developed in order to ease the typesetting of
% requirements documents in \LaTeX{}.
% 
% ^^A %%%%% Usage %%%%%%
% \section{Usage}
% 
% ^^A %%% Package option |compact| %%%
% Just like any other package, you need to request this package with a
% |\usepackage| command in the preamble.  This package take some
% options on how requirements are presented, either normally (within
% some bounding box) or compactly (as a compact description list).  
% 
% So in the simpler case, one just types
% 
% \indent |\usepackage{reqdoc}|
% 
% \noindent to load the package with the default option
% (i.e. display requirements in normal size), or uses
% 
% \indent |\usepackage[compact]{reqdoc}|
% 
% \noindent when the user wants to present the requirements documents
% in the compact form.
% 
% The rest of this section is to give descriptions of the main
% environment |requirements| and macros for typesetting different
% types of requirements.
% 
% ^^A %%% Environment |requirements| %%%
% \DescribeEnv{requirements}
% The requirements should be put within the |requirements|
% environment.  The environment will be adapted accordingly to the
% package option, e.g. |compact|.
% 
% ^^A %%% Requirement macros %%%
% \DescribeMacro{\req}
% \DescribeMacro{\eqp}
% \DescribeMacro{\fun}
% \DescribeMacro{\sys}
% \DescribeMacro{\saf}
% \DescribeMacro{\env}
% \DescribeMacro{\asm}
% \DescribeMacro{\sched}
% \DescribeMacro{\alg}
% To typeset a requirement, several macros are provided for different
% type of requirements. The signature of the requirement macros are as
% follows.
%
% \indent |\macroname|\oarg{width}\marg{ref-label}\marg{requirement}
% 
% \noindent The optional argument \oarg{width} defines the width for
% the requirement box in the default normal mode. This option is
% ignored in the compact mode.  The default value for this optional
% argument is |0.6\textwidth|.  The argument \marg{ref-label} defines
% the reference label, by which the requirement can be referenced
% later using |\ref{ref-label}|.  Note that \marg{ref-label} is the
% LaTeX label for referencing, which is different from the
% automatically generated requirement label (discussed below).
% Finally the argument |\marg{requirement}| is some concise
% description test.
%
% ^^A %%% Requirement counter requirement@counter %%%
% \DescribeMacro{\requirement@counter}
% Each requirement are labelled automatically, where the labels
% consist of some prefix (depending on the type of the requirement),
% and some sequence number.  The requirement sequence number are
% controlled by a counter |requirement@counter| which is advanced
% automatically after each requirement declaration.
%
% Table~\ref{tab:req-macros} lists the macros, their intended purpose,
% and the prefixes.  The convention for different types of
% requirements does not mean to be exhaustive or necessarily disjoint. 
% \begin{table}[!htbp]
%  \centering
%  \begin{tabular}{c@{\quad\quad}c@{\quad\quad}c}
%    \hline
%    Macro & Purpose & Prefix \\
%    \hline
%    |\req| & General requirements & |REQ| \\
%    |\eqp| & Equipment requirements & |EQP| \\
%    |\fun| & Functional requirements & |FUN| \\
%    |\sys| & System requirements & |SYS| \\
%    |\saf| & Safety requirements & |SAF| \\
%    |\env| & Environment requirements & |ENV| \\
%    |\asm| & Assumptions & |ASM| \\
%    |\sched| & Scheduling requirements & |SCHED| \\
%    |\alg| & Algorithmic requirements & |ALG| \\
%    \hline
%  \end{tabular}
%  \caption{List of macros for requirements}
%  \label{tab:req-macros}
% \end{table}
%
% ^^A %%% Requirement spacing macro \ReqSpacing %%%
% \DescribeMacro{\ReqSpacing}
% An useful macro is |\ReqSpacing| to give some spacing between
% requirements within the |requirements| environment.  This spacing
% will be ignored when the package option |compact| is enabled.  The
% signature of the command is as follows.
% 
% \indent |\ReqSpacing|\oarg{height}
%
% \noindent where the optional argument \oarg{height} define the
% spacing gap (default |2ex|).
%
% ^^A %%% Example %%%
% Below is a some sample requirements typeset using the newly defined
% environment and macros.
% \begin{verbatim}
% \begin{requirements}
%   \asm[0.5\textwidth]{asm:instructors}
%   {Instructors are members of the club.}
%   \ReqSpacing
%   \asm[0.5\textwidth]{asm:participants}
%   {Participants are members of the club.}
%   \ReqSpacing[4ex]
%   \req{req:course-status}
%   {A course is either \emph{opened} or \emph{closed}.}
%   \ReqSpacing
%   \req{req:open-course}{The system allows to open a closed course.}
%   \ReqSpacing
%   \req{req:close-course}{The system allows to close an opened course.}
% \end{requirements}
% \end{verbatim}
% 
% In the default normal mode, the result looks like the following
%
% \begin{requirementsbox}
%   \setcounter{asmc}{\value{requirement@counter}}
%   \requirementbox[0.5\textwidth]{ASM}
%   {Instructors are members of the club.}
%   \refstepcounter{asmc}
%   \label{asm:instructors}
%   \ReqSpacing
%   \setcounter{asmc}{\value{requirement@counter}}
%   \requirementbox[0.5\textwidth]{ASM}
%   {Participants are members of the club.}
%   \refstepcounter{asmc}
%   \label{asm:participants}
%   \ReqSpacing[4ex]
%   \setcounter{reqc}{\value{requirement@counter}}
%   \requirementbox{REQ}
%   {A course is either \emph{opened} or \emph{closed}.}
%   \refstepcounter{reqc}
%   \label{req:course-status}
%   \ReqSpacing
%   \setcounter{reqc}{\value{requirement@counter}}
%   \requirementbox{REQ}{The system allows to open a closed course.}
%   \refstepcounter{reqc}
%   \label{req:open-course}
%   \ReqSpacing
%   \setcounter{reqc}{\value{requirement@counter}}
%   \requirementbox{REQ}{The system allows to close an opened course.}
%   \refstepcounter{reqc}
%   \label{req:close-course}
% \end{requirementsbox}
%
% When the package option |compact| is enabled. the result looks as
% follows.
%
% \setcounter{requirement@counter}{0}
% \begin{requirementscompact}
%   \setcounter{asmc}{\value{requirement@counter}}
%   \requirementcompact{ASM}{Instructors are members of the club.}
%   \refstepcounter{asmc}
%   \setcounter{asmc}{\value{requirement@counter}}
%   \requirementcompact{ASM}{Participants are members of the club.}
%   \refstepcounter{asmc}
%   \setcounter{reqc}{\value{requirement@counter}}
%   \requirementcompact{REQ}{A course is either \emph{opened} or
%   \emph{closed}.}
%   \refstepcounter{reqc}
%   \setcounter{reqc}{\value{requirement@counter}}
%   \requirementcompact{REQ}{The system allows to open a closed
%   course.}
%   \refstepcounter{reqc}
%   \setcounter{reqc}{\value{requirement@counter}}
%   \requirementcompact{REQ}{The system allows to close an opened
%   course.}
%   \refstepcounter{reqc}
% \end{requirementscompact}
%
% The requirements can be referenced using the previously defined
% labels.  An example is as follows.
% 
% \begin{center}
%   \begin{minipage}[c]{0.65\linewidth}
% \begin{verbatim}
% There are two assumptions
% \ref{asm:instructors}, \ref{asm:participants},
% and three requirements \ref{req:course-status},
% \ref{req:open-course}, \ref{req:close-course}.
% \end{verbatim}
%   \end{minipage}
%   \hspace{1em}
%   \begin{minipage}[c]{0.3\linewidth}
%     There are two assumptions \ref{asm:instructors},
%     \ref{asm:participants}, and three requirements
%     \ref{req:course-status}, \ref{req:open-course}, \ref{req:close-course}.
%   \end{minipage}
% \end{center}
%
% ^^A %%%%% Implementation %%%%%
% \section{Implementation}
%
% ^^A %%% Package loading %%% 
% The implementation is quite straightforward.  We first request the
% |varioref| package for referencing the requirements later.
% 
% \iffalse ^^A BEGIN Produce comments only in the resulting style file
%<reqdoc>
%<reqdoc>%%%%% BEGIN Package loading %%%%%
% \fi ^^A END Produce comments only in the resulting style file
%
%    \begin{macrocode}
\RequirePackage{varioref}
%    \end{macrocode}
%
% \iffalse ^^A BEGIN Produce comments only in the resulting style file
%<reqdoc>%%%%% END Package loading %%%%%
%<reqdoc>
% \fi ^^A END Produce comments only in the resulting style file
% 
% ^^A %%% Package option |compact| %%%
% Subsequently, we declare the option \oarg{compact} and redefine the
% \textsf{requirements} environment accordingly.
% \changes{v1.2}{2012/06/27}{Package option \oarg{compact} added}.
% For the environment and macros that depending on the package
% options, we define different auxiliary versions of the environment
% and macros, and use the appropriate version depending on the
% package option.  In our case, the environment |requirements|, macros
% |\requirement| and |\ReqSpacing| are depended on the package option.
% 
% \iffalse ^^A BEGIN Produce comments only in the resulting style file
%<reqdoc>
%<reqdoc>%%%%% BEGIN Declaration of options %%%%%
% \fi^^A END Produce comments only in the resulting style file
%
% \begin{environment}{requirements}
%   \changes{v1.3}{2012/06/29}{Environment modified}
%   By default |requirements| environment is the same as
%   |requirementsbox| environment (to be defined later).
%    \begin{macrocode}
\newenvironment{requirements}
  {\begin{requirementsbox}}
  {\end{requirementsbox}}
%    \end{macrocode}
% \end{environment}^^A{requirements}
%
% \begin{macro}{\ReqSpacing}
%   \changes{v1.2}{2012/06/27}{Macro added}
%   \changes{v1.3}{2012/06/29}{Macro modified}
%   By default |\ReqSpacing| is the same as |\ReqSpacingBox|.
%    \begin{macrocode}
\newcommand{\ReqSpacing}[1][2ex]{\ReqSpacingBox[#1]}
%    \end{macrocode}
% \end{macro}^^A{\ReqSpacing}
%
% \begin{macro}{\requirement}
%   \changes{v1.3}{2012/06/29}{Macro modified}
%   By default |\requirement| is the same as |\requirementbox|.
%    \begin{macrocode}
\newcommand{\requirement}[3][0.6\textwidth]{
  \requirementbox[#1]{#2}{#3}
}
%    \end{macrocode}
% \end{macro}^^A{\requirement}
%
% By declaring option |compact|, the previously defined environment
% and macros are redefined accordingly.  In particular |\ReqSpacing|
% is simply ignored.  These environment and macros are for
% implementation purpose only. The user should not use them directly.
%    \begin{macrocode}
\DeclareOption{compact}{
  \renewenvironment{requirements}
    {\begin{requirementscompact}}
    {\end{requirementscompact}}

  \renewcommand{\requirement}[3][0.6\textwidth]{
    \requirementcompact{#2}{#3}
  }

  \renewcommand{\ReqSpacing}[1][]{}
}
%    \end{macrocode}
%
% \iffalse^^A BEGIN Produce comments only in the resulting style file
%<reqdoc>%%%%% END Declaration of options %%%%%
% \fi^^A END Produce comments only in the resulting style file
%
% Afterwards, we process the options accordingly
%
% \iffalse^^A BEGIN Produce comments only in the resulting style file
%<reqdoc>
%<reqdoc>%%%%% BEGIN Execution of options %%%%%
% \fi^^A END Produce comments only in the resulting style file
% 
%    \begin{macrocode}
\ProcessOptions
%    \end{macrocode}
%
% \iffalse^^A BEGIN Produce comments only in the resulting style file
%<reqdoc>%%%%% END Execution of options %%%%%
% \fi^^A END Produce comments only in the resulting style file
%
% In the subsequent, we define the environment and macro for different
% package options.
% \begin{environment}{requirementsbox}
%   \changes{v1.3}{2012/06/28}{Environment added}
%   We define environment |requirementsbox| is the same as
%   environment |center|.
% 
% \iffalse^^A BEGIN Produce comments only in the resulting style file
%<*reqdoc>

%%%%% Utility environment for requirements in boxes.
%%%%% Note: Users should not use this environment directly
%</reqdoc>
% \fi^^A END Produce comments only in the resulting style file
%
%    \begin{macrocode}
\newenvironment{requirementsbox}
  {\begin{center}}
  {\end{center}}

%    \end{macrocode}
% \end{environment}^^A{requirementsbox}
%
% \begin{environment}{requirementscompact}
%   \changes{v1.3}{2012/06/28}{Environment added}
%   We define environment |requirementcompact| is the same as
%   environment |description| with font |\small|
%
% \iffalse^^A BEGIN Produce comments only in the resulting style file
%<*reqdoc>

%%%%% Utility environment for requirements as descriptions.
%%%%% Note: Users should not use this environment directly
%</reqdoc>
% \fi^^A END Produce comments only in the resulting style file
%    \begin{macrocode}
  \newenvironment{requirementscompact}
    {\begin{description}\small}
    {\end{description}}
%    \end{macrocode}
% \end{environment}^^A{requirementscompact}
%
%
% \begin{macro}{\ReqSpacingBox}
%   \changes{v1.3}{2012/06/29}{Macro added}
%   We define |\ReqSpacingBox| by passing the optional argument
%   \oarg{height} to |\\|.
%
% \iffalse^^A BEGIN Produce comments only in the resulting style file
%<*reqdoc>

%%%%% Utility macro for spacing between requirements.
%%%%% Note: Users should not use this environment directly
%</reqdoc>
% \fi^^A END Produce comments only in the resulting style file
%
%    \begin{macrocode}
\newcommand{\ReqSpacingBox}[1][2ex]{\\[#1]}
%    \end{macrocode}
% \end{macro}^^A{\ReqSpacingBox}
%
% \begin{macro}{\requirementbox}
%   \changes{v1.3}{2012/06/29}{Macro added}
%   The macro for typesetting a requirement in a box is as follows.
% 
% \iffalse^^A BEGIN Produce comments only in the resulting style file
%<*reqdoc>

%%%%% Requirement in a box: 
%%%%%   Provide a requirement box with label and description text.
%%%%%   The requirement counter is used automatically.
%%%%% Note:
%%%%%   Users should use the sub-category requirements, e.g. \fun, \env
%%%%%   This macros is used for defining those specific requirements
%%%%%
%%%%% Arguments:
%%%%%   1. (Optional) Width of the explanation text. 
%%%%%      (Default value: 0.6\textwidth)
%%%%%   2. Label text prefix (e.g. ENV, FUN, SAF, ASM, EQP, etc.)
%%%%%   3. Explanation text.
%%%%%
%%%%% Usage: 
%%%%%   \requirementbox[0.8\textwidth]{ASM}{An assumption is ...}
%%%%%   \requirementbox{FUN}{A functional requirement is ...}
%</reqdoc>
% \fi^^A END Produce comments only in the resulting style file
%
%    \begin{macrocode}
\newcommand{\requirementbox}[3][0.6\textwidth]{
  \medskip
  \refstepcounter{requirement@counter}
  \begin{tabular}{|@{\quad}c@{\quad}|@{\quad}c@{\quad}|}
    \hline
    & \\
    \textsf{#2~\arabic{requirement@counter}} &
    \begin{minipage}[c]{#1}
      \begin{center}
        #3
      \end{center}
    \end{minipage} \\
    & \\
    \hline
  \end{tabular}
}
%    \end{macrocode}
% \end{macro}^^A{\requirementbox}
%
% \begin{macro}{\requirementcompact}
%   \changes{v1.3}{2012/06/29}{Macro added}
%   The macro for typesetting a requirement as a compact item is as follows.
%    \begin{macrocode}
  \newcommand{\requirementcompact}[2]{
    \medskip
    \refstepcounter{requirement@counter}
    \item[\textsf{#1~\arabic{requirement@counter}}] #2
  }
%    \end{macrocode}
% \end{macro}^^A{\requirementcompact}
% 
% \begin{macro}{requirement@counter}
% We define a counter for requirements, which will be used within the
% the labels of the requirements.
% 
% \iffalse^^A BEGIN Produce comments only in the resulting style file
%<reqdoc>
%<reqdoc>%%%%% The shared counter for requirements
% \fi^^A END Produce comments only in the resulting style file
%
%    \begin{macrocode}
\newcounter{requirement@counter}
\labelformat{requirement@counter}{REQ~#1}
%    \end{macrocode}
% \end{macro}^^A{requirement@counter}
%
%
% We define some specific commands for different type of requirements.
% These commands are those that the user should use to typeset the
% requirements.  Each command associated with a new counter for each
% type of requirements and a unique label format.  Each type of
% requirement use the underlying macro |\requirement|
% as defined earlier.
%
% \begin{macro}{reqc}
% \begin{macro}{\req}
%   \changes{v1.2}{2011/09/22}{Macro added}
%   Command for typesetting REQUIREMENT (typically prefixed with REQ).
%
% \iffalse^^A BEGIN Produce comments only in the resulting style file
%<*reqdoc>

%%%%% Specific command for REQUIREMENT (REQ). This command used the
%%%%% general macro \requirement for requirement.
%%%%% Arguments:
%%%%%   1. (Optional) Width of the explanation text. 
%%%%%      (Default value: 0.6\textwidth)
%%%%%   2. Reference label.
%%%%%   3. Explanation text.
%%%%%
%%%%% Usage: 
%%%%%   \req[0.8\textwidth]{req1}{The 1st requirement is ...}
%%%%%   \req{req2}{The 2nd requirement is ...}
%%%%%   Requirements \ref{req1} and \ref{req2} are modelled ...
%</reqdoc>
% \fi^^A END Produce comments only in the resulting style file
%
%    \begin{macrocode}
\newcounter{reqc}
\labelformat{reqc}{REQ~#1}
\newcommand{\req}[3][0.6\textwidth]{
  \setcounter{reqc}{\value{requirement@counter}}
  \requirement[#1]{REQ}{#3}
  \refstepcounter{reqc}
  \label{#2}
}
%    \end{macrocode}
% \end{macro}^^A{\req}
% \end{macro}^^A{reqc}
%
% \begin{macro}{eqpc}
% \begin{macro}{\eqp}
%   Command for typesetting EQUIPMENT requirements (typically prefixed with EQP).
%
% \iffalse^^A BEGIN Produce comments only in the resulting style file
%<*reqdoc>

%%%%% Specific command for EQUIPMENT (EQP) requirements. This command
%%%%% used the general macro \requirement for requirement.
%%%%% Arguments:
%%%%%   1. (Optional) Width of the explanation text. 
%%%%%      (Default value: 0.6\textwidth)
%%%%%   2. Reference label.
%%%%%   3. Explanation text.
%%%%%
%%%%% Usage: 
%%%%%   \eqp[0.8\textwidth]{eqp1}{The 1st equipment is ...}
%%%%%   \eqp{eqp2}{The 2nd equipment is ...}
%%%%%   Requirements \ref{eqp1} and \ref{eqp2} are modelled ...
%</reqdoc>
% \fi^^A END Produce comments only in the resulting style file
%
%    \begin{macrocode}
\newcounter{eqpc}
\labelformat{eqpc}{EQP~#1}
\newcommand{\eqp}[3][0.6\textwidth]{
  \setcounter{eqpc}{\value{requirement@counter}}
  \requirement[#1]{EQP}{#3}
  \refstepcounter{eqpc}
  \label{#2}
}
%    \end{macrocode}
% \end{macro}^^A{\eqp}
% \end{macro}^^A{eqpc}
%
% \begin{macro}{func}
% \begin{macro}{\fun}
%   Command for typesetting FUNCTIONAL requirements (typically
%   prefixed with FUN).
%
% \iffalse^^A BEGIN Produce comments only in the resulting style file
%<*reqdoc>

%%%%% Specific command for FUNCTIONAL (FUN) requirements. This command
%%%%% used the general macro \requirement for requirement.
%%%%% Arguments:
%%%%%   1. (Optional) Width of the explanation text. 
%%%%%      (Default value: 0.6\textwidth)
%%%%%   2. Reference label.
%%%%%   3. Explanation text.
%%%%%
%%%%% Usage: 
%%%%%   \fun[0.8\textwidth]{fun1}{The 1st functional requirement is ...}
%%%%%   \fun{fun2}{The 2nd functional requirement is ...}
%%%%%   Requirements \ref{fun1} and \ref{fun2} are modelled ...
%</reqdoc>
% \fi^^A END Produce comments only in the resulting style file
%
%    \begin{macrocode}
\newcounter{func}
\labelformat{func}{FUN~#1}
\newcommand{\fun}[3][0.6\textwidth]{
  \setcounter{func}{\value{requirement@counter}}
  \requirement[#1]{FUN}{#3}
  \refstepcounter{func}
  \label{#2}
}
%    \end{macrocode}
% \end{macro}^^A{\fun}
% \end{macro}^^A{func}
%
% \begin{macro}{sysc}
% \begin{macro}{\sys}
%   Command for typesetting SYSTEM requirements (typically
%   prefixed with SYS).
%
% \iffalse^^A BEGIN Produce comments only in the resulting style file
%<*reqdoc>

%%%%% Specific command for SYSTEM (SYS) requirements. This command
%%%%% used the general macro \requirement for requirement.
%%%%% Arguments:
%%%%%   1. (Optional) Width of the explanation text. 
%%%%%      (Default value: 0.6\textwidth)
%%%%%   2. Reference label.
%%%%%   3. Explanation text.
%%%%%
%%%%% Usage: 
%%%%%   \sys[0.8\textwidth]{sys1}{The 1st system requirement is ...}
%%%%%   \sys{sys2}{The 2nd system requirement is ...}
%%%%%   Requirements \ref{sys1} and \ref{sys2} are modelled ...
%</reqdoc>
% \fi^^A END Produce comments only in the resulting style file
%
%    \begin{macrocode}
\newcounter{sysc}
\labelformat{sysc}{SYS~#1}
\newcommand{\sys}[3][0.6\textwidth]{
  \setcounter{sysc}{\value{requirement@counter}}
  \requirement[#1]{SYS}{#3}
  \refstepcounter{sysc}
  \label{#2}
}
%    \end{macrocode}
% \end{macro}^^A{\sys}
% \end{macro}^^A{sysc}
%
% \begin{macro}{safc}
% \begin{macro}{\saf}
%   Command for typesetting SAFETY requirements (typically
%   prefixed with SAF).
%
% \iffalse^^A BEGIN Produce comments only in the resulting style file
%<*reqdoc>

%%%%% Specific command for SAFETY (SAF) requirements. This command
%%%%% used the general macro \requirement for requirement.
%%%%% Arguments:
%%%%%   1. (Optional) Width of the explanation text. 
%%%%%      (Default value: 0.6\textwidth)
%%%%%   2. Reference label.
%%%%%   3. Explanation text.
%%%%%
%%%%% Usage: 
%%%%%   \saf[0.8\textwidth]{saf1}{The 1st safety requirement is ...}
%%%%%   \saf{saf2}{The 2nd safety requirement is ...}
%%%%%   Requirements \ref{saf1} and \ref{saf2} are modelled ...
%</reqdoc>
% \fi^^A END Produce comments only in the resulting style file
%
%    \begin{macrocode}
\newcounter{safc}
\labelformat{safc}{SAF~#1}
\newcommand{\saf}[3][0.6\textwidth]{
  \setcounter{safc}{\value{requirement@counter}}
  \requirement[#1]{SAF}{#3}
  \refstepcounter{safc}
  \label{#2}
}
%    \end{macrocode}
% \end{macro}^^A{\saf}
% \end{macro}^^A{safc}
%
% \begin{macro}{envc}
% \begin{macro}{\env}
%   Command for typesetting ENVIRONMENT requirements (typically
%   prefixed with ENV).
%
% \iffalse^^A BEGIN Produce comments only in the resulting style file
%<*reqdoc>

%%%%% Specific command for ENVIRONMENT (ENV) requirements. This command
%%%%% used the general macro \requirement for requirement.
%%%%% Arguments:
%%%%%   1. (Optional) Width of the explanation text. 
%%%%%      (Default value: 0.6\textwidth)
%%%%%   2. Reference label.
%%%%%   3. Explanation text.
%%%%%
%%%%% Usage: 
%%%%%   \env[0.8\textwidth]{env1}{The 1st environment requirement is ...}
%%%%%   \env{env2}{The 2nd environment requirement is ...}
%%%%%   Requirements \ref{env1} and \ref{env2} are modelled ...
%</reqdoc>
% \fi^^A END Produce comments only in the resulting style file
%
%    \begin{macrocode}
\newcounter{envc}
\labelformat{envc}{ENV~#1}
\newcommand{\env}[3][0.6\textwidth]{
  \setcounter{envc}{\value{requirement@counter}}
  \requirement[#1]{ENV}{#3}
  \refstepcounter{envc}
  \label{#2}
}
%    \end{macrocode}
% \end{macro}^^A{\env}
% \end{macro}^^A{envc}
%
% \begin{macro}{asmc}
% \begin{macro}{\asm}
%   Command for typesetting ASSUMPTIONS (typically
%   prefixed with ASM).
%
% \iffalse^^A BEGIN Produce comments only in the resulting style file
%<*reqdoc>

%%%%% Specific command for ASSUMPTIONS (ASM). This command
%%%%% used the general macro \requirement for requirement.
%%%%% Arguments:
%%%%%   1. (Optional) Width of the explanation text. 
%%%%%      (Default value: 0.6\textwidth)
%%%%%   2. Reference label.
%%%%%   3. Explanation text.
%%%%%
%%%%% Usage: 
%%%%%   \asm[0.8\textwidth]{asm1}{The 1st assumption is ...}
%%%%%   \asm{asm2}{The 2nd assumption is ...}
%%%%%   Requirements \ref{asm1} and \ref{asm2} are modelled ...
%</reqdoc>
% \fi^^A END Produce comments only in the resulting style file
%
%    \begin{macrocode}
\newcounter{asmc}
\labelformat{asmc}{ASM~#1}
\newcommand{\asm}[3][0.6\textwidth]{
  \setcounter{asmc}{\value{requirement@counter}}
  \requirement[#1]{ASM}{#3}
  \refstepcounter{asmc}
  \label{#2}
}
%    \end{macrocode}
% \end{macro}^^A{\asm}
% \end{macro}^^A{asmc}
%
% \begin{macro}{schedc}
% \begin{macro}{\sched}
%   Command for typesetting SCHEDULING requirements (typically
%   prefixed with SCHED).
%
% \iffalse^^A BEGIN Produce comments only in the resulting style file
%<*reqdoc>

%%%%% Specific command for SCHEDULING (SCHED) requirements. This command
%%%%% used the general macro \requirement for requirement.
%%%%% Arguments:
%%%%%   1. (Optional) Width of the explanation text. 
%%%%%      (Default value: 0.6\textwidth)
%%%%%   2. Reference label.
%%%%%   3. Explanation text.
%%%%%
%%%%% Usage: 
%%%%%   \sched[0.8\textwidth]{sched1}{The 1st scheduling requirement is ...}
%%%%%   \sched{sched2}{The 2nd scheduling requirement is ...}
%%%%%   Requirements \ref{sched1} and \ref{sched2} are modelled ...
%</reqdoc>
% \fi^^A END Produce comments only in the resulting style file
%
%    \begin{macrocode}
\newcounter{schedc}
\labelformat{schedc}{SCHED~#1}
\newcommand{\sched}[3][0.6\textwidth]{
  \setcounter{schedc}{\value{requirement@counter}}
  \requirement[#1]{SCHED}{#3}
  \refstepcounter{schedc}
  \label{#2}
}
%    \end{macrocode}
% \end{macro}^^A{\sched}
% \end{macro}^^A{schedc}
%
% \begin{macro}{algc}
% \begin{macro}{\alg}
%   Command for typesetting ALGORITHM requirements (typically
%   prefixed with ALG).
%
% \iffalse^^A BEGIN Produce comments only in the resulting style file
%<*reqdoc>

%%%%% Specific command for ALGORITHM (ALG) requirements. This command
%%%%% used the general macro \requirement for requirement.
%%%%% Arguments:
%%%%%   1. (Optional) Width of the explanation text. 
%%%%%      (Default value: 0.6\textwidth)
%%%%%   2. Reference label.
%%%%%   3. Explanation text.
%%%%%
%%%%% Usage: 
%%%%%   \alg[0.8\textwidth]{alg1}{The 1st algorithm requirement is ...}
%%%%%   \alg{alg2}{The 2nd algorithm requirement is ...}
%%%%%   Requirements \ref{alg1} and \ref{alg2} are modelled ...
%</reqdoc>
% \fi^^A END Produce comments only in the resulting style file
%
%    \begin{macrocode}
\newcounter{algc}
\labelformat{algc}{ALG~#1}
\newcommand{\alg}[3][0.6\textwidth]{
  \setcounter{algc}{\value{requirement@counter}}
  \requirement[#1]{ALG}{#3}
  \refstepcounter{algc}
  \label{#2}
}
%    \end{macrocode}
% \end{macro}^^A{\alg}
% \end{macro}^^A{algc}
%
% \Finale
% \PrintChanges
% \PrintIndex