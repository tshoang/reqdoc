\title{A Requirements Document \\for a Course Management System}
\author{Thai Son Hoang\\ETH Zurich\\\texttt{<htson at inf dot ethz dot
    ch>}}

\begin{document}
\maketitle

We present below a sample requirements document and a refinement
strategy that refers to the informal requirements.

\section{Requirements Document}
\label{sec:requ-docum}

A club has some fixed \emph{members}, amongst
them are \emph{instructors} and \emph{participants}.  Note that a
member can be both an instructor and a participant.
\begin{requirements}
  \asm[0.5\textwidth]{asm:instructors}{Instructors are members of the club.}
  \ReqSpacing
  \asm[0.5\textwidth]{asm:participants}{Participants are members of the club.}
\end{requirements}

There are pre-defined \emph{courses} that can be offered by a club.
Each course is associated with exactly one fixed instructor.
\begin{requirements}
  \asm[0.5\textwidth]{asm:courses}{There are pre-defined courses.}
  \ReqSpacing
  \asm{asm:course-instructor}{Each course is assigned to one fixed instructor.}
\end{requirements}
A course is either \emph{opened} or \emph{closed} and is managed by
the system.
\begin{requirements}
  \req{req:course-status}{A course is either \emph{opened} or
    \emph{closed}.}
  \ReqSpacing
  \req{req:open-course}{The system allows to open a closed course.}
  \ReqSpacing
  \req{req:close-course}{The system allows to close an opened course.}
\end{requirements}

The number of opened courses is limited.
\begin{requirements}
  \req[0.5\textwidth]{req:course-limit}{The number of opened courses \\cannot exceed a given limit.}
\end{requirements}

Only when a course is opened, participants can \emph{register} for the
course.  An important constraint for registration is that an
instructor cannot attend his own courses.
\begin{requirements}
  \req[0.65\textwidth]{req:registration}{Participants can only register for an opened
    course.}
  \ReqSpacing
  \req{req:conflict}{Instructors cannot attend their own courses.}
\end{requirements}

\section{A Refinement Strategy}
\label{sec:refinement-strategy}

Typically, before attempting to model a system, it is important to
sketch a plan of how the requirements are going to be addressed.
These plans are called \emph{refinement strategies}.  For our running
example, we adopt the following strategy.
\begin{description}
\item[Initial Model] To model how courses are opened and closed by the
  system, limiting the number of opened courses (\ref{asm:courses},
  \ref{req:course-status}, \ref{req:open-course}, \ref{req:close-course}, and
  \ref{req:course-limit}).
\item[First Refinement] To model the club members, including
  instructors and participants, along with registration for courses
  (\ref{asm:instructors}, \ref{asm:participants},
  \ref{asm:course-instructor}, \ref{req:registration}, and
  \ref{req:conflict}).
\item[Second Refinement] To data refine the model from the first refinement.
\end{description}

\end{document}